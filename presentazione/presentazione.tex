\documentclass[aspectratio=43]{beamer}

\usetheme{Padova}
\usepackage[utf8x]{inputenc}
\usepackage{listings}
\usepackage{wrapfig}

\usepackage{tikz,xparse}
\usetikzlibrary{decorations.pathreplacing,}
\newcommand{\tikzmark}[1]{\tikz[baseline={(#1.base)},overlay,remember picture] \node[outer sep=0pt, inner sep=0pt] (#1) {\phantom{A}};}

%% syntax
%%%\mybrace{<first>}{<second>}[<Optional text>]
\NewDocumentCommand\mybrace{mmo}{%
\IfValueTF {#3}{%
\begin{tikzpicture}[overlay, remember picture,decoration={brace,amplitude=1ex}]
  \draw[decorate,thick] (#1.north east) -- (#2.south east) node[midway, right=0.1cm] {$=$}node[midway, right=0.5cm,text=grigioPantano,text width = 1.5in,] {{#3}};
\end{tikzpicture}%
}%
{%
\begin{tikzpicture}[overlay, remember picture,decoration={brace,amplitude=1ex}]
  \draw[decorate,thick] (#1.north east) -- (#2.south east);
\end{tikzpicture}%
}%
}%

\usepackage{subfigure}

\title{PICAT \\Analisi del linguaggio}
\author{Marco Zanella - 1155185}
\date{22 marzo 2017}


\begin{document}
	\maketitle
	\iffalse
PICAT è un linguaggio di programmazione creato da un professore di informatica Neng-Fa Zhou di Brooklyn e il nome è una sigla che significa Pattern-matching Intuitive Constraint Actors Tabling
* Pattern Matching perchè il linguaggio permette di definire funzioni o predicati utilizzando appunto il pattern matching
* Intuitive perchè secondo l'ideatore è stato creato cercando di avvicinare il più possibile il linguaggio a quello che si vuole modellare, anche grazi l'introduzione di costrutti di vari tipi di programmazione come i loop della programmazione imperativa oppure la list comprehension tipica dei linguaggi funzionali
* Constaint perchè supporta la programmazione con vincoli e poichè prevede 3 moduli chiamati cp, sat e mip per la soluzione di questo tipo di problemi
* Actor perchè permette la programmazione event driven ***COME SI FA CHE NON HO TROVATO MATERIALE DECENTE?***
* Tabling perchè il linguaggio ha delle primitive che permettono di effettuare la memoizzazione di risultati senza dover ricalcolarli ogni volta
Come si capisce anche dal significato del nome questo linguaggio è stato influenzato da differenti altri linguaggi, supporta differenti paradigmi di programmazione
È stato modellato per essere un linguaggio di scripting utile soprattutto in problemi di intelligenza artificiale, per la risoluzione di sistemi di vincoli, problemi di pianificazione o programmazione dinamica
È un linguaggio abbastanza recente(2013 e l'ultima release è la versione 2.1 di quest'anno) e poco conosciuto tant'è che ha una entry su wikipedia solamente in italiano e che nelle prime 100 posizioni dell'indice TIOBE non è presente
\fi

\begin{frame}{Informazioni generali}

	\begin{columns}
		\begin{column}{0.8\textwidth}
			\begin{itemize}
				\item Pattern-matching Intuitive Constraints Actors Tabling
				\item Creato da Neng-Fa Zhou, professore di informatica del Brooklyn College
				\item Pensato per risolvere problemi di intelligenza artificiale
					\begin{itemize}
						\item Influenzato da differenti linguaggi
						\item Implementa differenti paradigmi di programmazione
					\end{itemize}
			\end{itemize}
		\end{column}

		\begin{column}{0.3\textwidth}
			
			\begin{figure}
				\centering
				\includegraphics[scale=2.5]{res/picatLogo}
			\end{figure}

			\begin{figure}
				\centering
				\includegraphics[scale=0.22]{res/zhou}
			\end{figure}
		
		\end{column}
	\end{columns}

\end{frame}
	\input{slides/overview-tipi.tex}
	\input{slides/overview-examplesCompoundTypes.tex}
	\input{slides/overview-globalMaps.tex}
	\iffalse
Come già anticipato PICAT è un linguaggio multiparadigma, ed è stato influenzato da differenti linguaggi, sopratutto da Prolog. Inoltre è stato modellato come un linguaggi di scripting, prendendo anche come esempio Python.
\fi

\begin{frame}{Molte influenze}

	Implementa differenti paradigmi di programmazione
	\begin{itemize}
		\item Programmazione logica
		\item Programmazione imperativa
		\item Programmazione funzionale
	\end{itemize}

	\vspace{1em}
	Pensato come linguaggio di scripting

	\begin{figure}
		\centering
		\includegraphics[scale=0.4]{res/influenze}
	\end{figure}
\end{frame}
	\iffalse
Della programmazione logica troviamo delle caratterisctiche tipiche come la possibilità di definire le relazioni, la possibilità di definire predicati e di farne backtracking (quindi abbiamo il non determinismo) ed infine è presente l'operatore di unificazione.

Per questo paradigma ho preparato tre esempi di utilizzo di PICAT
1- il primo riguarda le relazioni, praticamente ho definito alcuni predicati e alcuni fatti e si può interrogare PICAT a vedere se certe sentenze sono vere o false
2- il secondo esempio riguarda la definizione di predicati che utilizzano o meno il backtracking. Qui ho riscritto la funzione member, che è una funzione built-in che praticamente permette di dire che se un certo elemento appartiene ad una certa lista. La prima utilizza il backtracking e funziona mentre la seconda no
3- il terzo esempio è un programma che calcola fibonacci con e senza l'utilizzo del tabling, effettuando una stampa ogni volta che si entra nella funzione. Le due funzioni sono identiche, tranne che una è preceduta dalla keyword table, che permette la memoizzazione mentre l'altra no. Come posssiamo vedere dall'output il numero di volte che si effettuano effettivamente dei calcoli nel caso in cui si utilizzi il tabling diminuisce notevolmente nel caso in cui la funzione venga richiamata più volte
\fi

\begin{frame}[fragile]{Programmazione logica}

	\begin{itemize}
		\item Predicati*
			\begin{itemize}
				\item Senza backtraking: \\\hspace{1cm} \texttt{Head, Cond $\Rightarrow$ Body}
				\item Con backtracking: \\\hspace{1cm} \texttt{Head, Cond ?$\Rightarrow$ Body}
			\end{itemize}
		\item Operatore di unificazione \\4 differenti comportamenti per \texttt{T1 = T2}
			\begin{itemize}
				\item T1 è una variabile;
				\item T2 è una variabile;
				\item T1 e T2 sono valori atomici;
				\item T1 e T2 sono strutture;
			\end{itemize}
		\item Tabling*
	\end{itemize}

\end{frame}
	\iffalse
PICAT come linguaggi è molto simile al Prolog e difatti la prima versione di questo linguaggio riutilizza molto del codice di B-Prolog, quindi ha senso confrontare questi due linguaggi.
Una delle differenze più importanti è che in PICAT il non determinismo deve essere previsto esplicitamente da chi scrive un determinato predicato, mentre in Prolog tutti i predicati prevedono implicitamente il backtracking (se si vuole controllare il backtracking è necessario utilizzare l'operatore cut).
Un'altra differenza è che in PICAT la scelta del predicato da chiamare non viene fatta attraverso l'operatore di unificazione come in Prolog, bensì utilizzando il pattern matching.
L'ultima importante è sui costrutti: infatti PICAT prevede costrutti tipici della programmazione imperativa e funzionale, come cicli, array e list comprehension,l'assegnazione, e ciò permette di esprimere più facilmente alcuni concetti
\fi

\begin{frame}[fragile]{PICAT vs Prolog}

	\begin{columns}

		\begin{column}{1\textwidth}

			\begin{itemize}
				\item Non determinismo esplicito
				\item Scelta del predicato avviene tramite pattern matching
				\item Presenza di più costrutti
					\begin{itemize}
						\item Cicli
						\item Array e list comprehension
						\item Assegnazione
						\item Funzioni
					\end{itemize}
			\end{itemize}

		\end{column}

		\begin{column}{0.1\textwidth}
			\begin{figure}
				\vspace*{4cm}
				\hspace*{-4.5cm}
				\includegraphics[scale=0.2]{res/prologLogo}
			\end{figure}
		\end{column}

	\end{columns}

\end{frame}


	\iffalse
Oltre alla programmazione logica, PICAT è pensato anche per implementare la programmazione imperativa. Di questa possiede alcuni caratteristiche tipiche come l'assegnazione, oppure costrutti che riguardano il control flow quindi l'if-then-else ed i cicli. 
\fi

\begin{frame}{Programmazione imperativa - 1}

	\begin{columns}
		\begin{column}{0.5\textwidth}
			\begin{itemize}
				\item Assegnazione e side-effects
				\item If-Then-Else
				\item Cicli
				\begin{itemize}
					\item foreach
					\item while
					\item do..while
				\end{itemize}
			\end{itemize}
		\end{column}

		\begin{column}{0.5\textwidth}
			
			\begin{figure}%
				\centering
				\subfigure{\includegraphics[scale=0.09]{res/cLogo}}\qquad
				\subfigure{\includegraphics[scale=0.24]{res/c++Logo}}\\
				\subfigure{\includegraphics[scale=0.03]{res/javaLogo}}%
			\end{figure}

		\end{column}
	\end{columns}

\end{frame}
	\iffalse
PICAT prevede costrutti di differenti paradigmi di programmazione anche perchè, quando è stato creato, l'autore aveva in mente che potesse essere un ponte tra i linguaggi imperativi nei quali il programmatore deve dire al computer cosa deve fare e i linguaggi dichiarativi, nei quali invece viene descritto cosa si vuole come risultato
\fi

\begin{frame}{Programmazione imperativa - 2}

	\textit{"The purpose of Picat is to bridge the gap between imperative languages, 
	which tell the computer how things should be done, and declarative 
	languages, which tell the computer what to do, without detailing how it 
	should be done"}

\end{frame}
	\iffalse
Qui possiamo vedere un esempio di programma in stile imperativo scritto in PICAT. Il programma inizia dal main con una lista vuota X e un numero I pari a zero. Poi viene fatto un wile inizializzando la lista e incrementando ad ogni ciclo I. Finito il ciclo viene richiamata la funzione prova modifica la lista.
Poi la lista viene riassegnata e ancora modificata con la funzione prova
\fi

\begin{frame}[fragile, shrink=20]{Programmazione imperativa - Esempio}

	\lstinputlisting{../examples/imperativeProgramming.pi}

\end{frame}
	\iffalse
PICAT inoltre prevede anche alcuni aspetti funzionali. A parte permettere la definizione di funzioni ricorsive, queste possono essere definite tramite l'utilizzo di pattern matching. Oltre a ciò PICAT prevede una serie di funzioni già pronte per la manipolazione delle liste ed una certa facilità a modificarle. PICAT prevede inoltre un limitato supporto alle funzioni higher order: non se ne possono creare di nuove infatti, come in un qualsiasi linguaggi funzionale, ma ne prevede un insieme già creato. L'uso di queste funzioni è scoraggiato nel caso in cui si voglia creare programmi efficienti poichè creano un overhead elevato
\fi

\begin{frame}{Programmazione funzionale}

	\begin{figure}
		\hfill
		\includegraphics[scale=0.1]{res/lambda}
	\end{figure}

	\begin{itemize}
		\item Funzioni ricorsive
		\item Pattern matching per la definizione di funzioni
		\item Facilità a manipolare le liste (funzioni built-in e comprehension)
		\item Limitato supporto alle funzioni higher-order
			\begin{itemize}
				\item \texttt{apply}, \texttt{call}, \texttt{foldl}, \texttt{map}...
			\end{itemize}
	\end{itemize}

	\begin{figure}
		\hspace*{-8cm}
		\includegraphics[scale=0.1]{res/functional}
	\end{figure}

\end{frame}
	\iffalse
Poichè PICAT presenta alcune funzionalità riconducibili ai linguaggi lo possiamo confrontare con un linguaggio prettamente funzionale come Haskell. Le analogie tra i due linguaggi sono praticamente tutti gli aspetti funzionali di PICAT ovvero: pattern matching, list comprehension, funzioni higher order che però in PICAT sono sconsigliate mentre in haskell ovviamente no ed in entrambi i linguaggi i side effects sono scoraggiati.
Le differenze sono che nel il primo è staticamente tipato a differenza del secondo: in haskell infatti quando viene eseguito un programma tutto ha un certo tipo mentre in PICAT è possibile riassegnare una certa variabile a tipi differenti.
Infine PICAT non prevede l'utilizzo della lazy evaluation, che permette per esempio ad haskell di gestire liste infinite, mentre PICAT no, infatti ogni termine deve essere completamente valutato per esempio prima di essere passato come parametro ad una funzione. Comunque è possibile in qualche modo simularla tramite il costrutto freeze che permette di posticipare la valutazione di un termine
\fi

\begin{frame}{PICAT vs Haskell - 1}
	
	\begin{columns}

		\begin{column}{1\textwidth}

			Analogie
			\begin{itemize}
				\item Funzioni definite con pattern matching
				\item Supporto funzioni higher order (call, apply, map, foldl...)
				\item List comprehension
				\item Side effect scoraggiati
			\end{itemize}

			Differenze
			\begin{itemize}
				\item Dinamicamente tipato
				\item No lazy evaluation
				\item L'utilizzo delle funzioni higher order è scoraggiato
			\end{itemize}

		\end{column}

		\begin{column}{0.1\textwidth}
			\begin{figure}
				\vspace*{-5cm}
				\hspace*{-2cm}
				\includegraphics[scale=0.1]{res/haskellLogo}
			\end{figure}
		\end{column}

	\end{columns}
	
\end{frame}


	\iffalse
Qui possiamo vedere un esempio di due programmi uno scritto in PICAT, l'altro in haskell, che ordinano delle liste utilizzando il quicksort. Come possiamo notare grazie al pattern matching e alla list comprehension i due codici sono pressochè uguali.
\fi

\begin{frame}[fragile, shrink=1]{PICAT vs Haskell - 2}
	
	Quicksort in Picat							
	\lstinputlisting{../examples/quicksort.pi}

	\vspace{1cm}
	
	Quicksort in Haskell
	\lstinputlisting{../examples/quicksort.hs}

\end{frame}


	\input{slides/linguaggioDiScripting.tex}
	\iffalse
Possiamo quindi confrontare PICAT con Python. Sono stati entrambi progettati per essere linguaggi di scripting, dinamicamente tipati e permettono side effects. Una delle prime differenze è che anche se previsti PICAT scoraggia l'utilizzo di quest'ultimi mentre in Python ci sono anche funzioni built-in che fanno side effects. Ci sono poi altre differenze tra questi due linguaggi: Python è object oriented a differenza di PICAT che al massimo prevede le strutture, è differente la gestione di liste e array ed inoltre quando un programma PICAT viene compilato, cicli e list comprehension sono trasformati in chiamate tail recoursive in modo da rendere il programma più efficiente, cosa che invece Python non fa
\fi

\begin{frame}{PICAT vs Python}

	\begin{columns}

		\begin{column}{1\textwidth}

			Analogie
			\begin{itemize}
				\item Entrambi pensati come linguaggi di scripting
				\item Dinamicamente tipati
				\item Side effects
			\end{itemize}

			Differenze
			\begin{itemize}
				\item In PICAT l'uso dei side effects è scoraggiato
				\item Python è object oriented a differenza di PICAT
				\item In Python ci sono solo array dinamici mentre in PICAT abbiamo liste e array (non dinamici)
				\item In Python non è supportata l'ottimizzazione con tail recursion
			\end{itemize}

		\end{column}

		\begin{column}{0.1\textwidth}
			\begin{figure}
				\vspace*{-5cm}
				\hspace*{-2cm}
				\includegraphics[scale=0.5]{res/pythonLogo}
			\end{figure}
		\end{column}

	\end{columns}


\end{frame}
	\begin{frame}{Riferimenti}

	\begin{itemize}
		\item Constraint solving and planning with Picat, N.F. Zhou, H. Kjellerstrand, J. Fruhman
		\item My First Look At Picat as a Modeling Language or Constraint Solving and Planning, H. Kjellerstrand
		\item Documentazione Picat
		\item Picat user's Guide
		\item http://www.hakank.org/picat/
		\item https://it.wikipedia.org/wiki/Picat
		\item http://picat-lang.org/
		\item Neng-Fa Zhou - Programming in Picat https://www.youtube.com/watch?v=pQi9IPr0MFY
	\end{itemize}
	
\end{frame}
\end{document}