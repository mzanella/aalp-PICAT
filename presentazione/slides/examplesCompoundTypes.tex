\iffalse
Qui possiamo vedere un po' di esempi per la creazione di array, liste, strutture e mappe
Liste e array possono essere creati elencando gli elementi, con new_list e new_array oppure con list e array comprehension
Per creare le strutture invece è necessario anteporre il dollaro prima del nome della struttura per far capire al sistema che non si tratta di una chiamata di funzione
le mappe si creano con la funzione new_map e poi è possibile inserire gli elementi con la funzione put e ricavarli invece con la funzione get. Nel caso venga richiesto il get da una mappa senza che tale chiave sia presente allora viene lanciata una eccezione, per evitarlo è possibile aggiungere un valore di default quando si fa la chiamata
\fi

\begin{frame}[fragile, shrink=1]{Creazione oggetti compound}
	
	\begin{lstlisting}
%Esempi per la creazione di una lista
Picat>X=[a,b,c],Y=new_list(3),Z=[Z:Z in 1..6,even(Z)]
X = [a,b,c]
Y = [_104c8,_104d8,_104e8]
Z = [2,4,6]
%Esempi per la creazione di un array
Picat>X={a,b,c},Y=new_array(3),Z={Z:Z in 1..6,even(Z)}
X = {a,b,c}
Y = {_1b220,_1b228,_1b230}
Z = {2,4,6}
%Esempio di struttura
Picat>X = $str(abc, 1000, a), Y=X[1]
X = str(abc,1000,a)
Y = abc
%Esempio con una mappa
Picat>X=new_map(),X.put(1, a),X.put(2, b),Y=X.get(1)
X = (map)[1 = a,2 = b]
Y = a
	\end{lstlisting}

\end{frame}