\iffalse
PICAT come linguaggi è molto simile al Prolog e difatti la prima versione di questo linguaggio riutilizza molto del codice di B-Prolog, quindi ha senso confrontare questi due linguaggi.
Una delle differenze più importanti è che in PICAT il non determinismo deve essere previsto esplicitamente da chi scrive un determinato predicato, mentre in Prolog tutti i predicati prevedono implicitamente il backtracking (se si vuole controllare il backtracking è necessario utilizzare l'operatore cut).
Un'altra differenza è che in PICAT la scelta del predicato da chiamare non viene fatta attraverso l'operatore di unificazione come in Prolog, bensì utilizzando il pattern matching.
L'ultima importante è sui costrutti: infatti PICAT prevede costrutti tipici della programmazione imperativa e funzionale, come cicli, array e list comprehension,l'assegnazione, e ciò permette di esprimere più facilmente alcuni concetti
\fi

\begin{frame}[fragile]{PICAT vs Prolog}

	\begin{columns}

		\begin{column}{1\textwidth}

			\begin{itemize}
				\item Non determinismo esplicito
				\item Scelta del predicato avviene tramite pattern matching
				\item Presenza di più costrutti
					\begin{itemize}
						\item Cicli
						\item Array e list comprehension
						\item Assegnazione
						\item Funzioni
					\end{itemize}
			\end{itemize}

		\end{column}

		\begin{column}{0.1\textwidth}
			\begin{figure}
				\vspace*{4cm}
				\hspace*{-4.5cm}
				\includegraphics[scale=0.2]{res/prologLogo}
			\end{figure}
		\end{column}

	\end{columns}

\end{frame}

