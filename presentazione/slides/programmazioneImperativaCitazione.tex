\iffalse
PICAT prevede costrutti di differenti paradigmi di programmazione anche perchè, quando è stato creato, l'autore aveva in mente che potesse essere un ponte tra i linguaggi imperativi nei quali il programmatore deve dire al computer cosa deve fare e i linguaggi dichiarativi, nei quali invece viene descritto cosa si vuole come risultato
\fi

\begin{frame}{Programmazione imperativa - 2}

	\textit{"The purpose of Picat is to bridge the gap between imperative languages, 
	which tell the computer how things should be done, and declarative 
	languages, which tell the computer what to do, without detailing how it 
	should be done"}

\end{frame}